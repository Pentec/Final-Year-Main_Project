
\subsubsection*{Description}
The PIMS should be accessible at almost all times, in particular during the peak operational hours of the hospital. This accessibility will be limited to the hospital network only, so as to ensure no information can be changed without approval.
		
\subsubsection*{Justification}
The reliability and availability requirements are very important seeing as the information to be kept on the system is highly valuable to the medical staff, as the need up-to-date information concerning the patients they are dealing with (lives could be at risk).							 
With this in mind, only a downtime of less than 2 hours, at most twice a month will be allowed, so as too allow for the medical staff to maximally use the system.
A high reliability rate is recommended to ensure that users do not encounter any errors and/or data corruption in their use of the system. The only leeway that will be given for errors, is to have at most one.
	
	
\subsubsection*{Mechanism}	
\begin{enumerate}
\item Strategy:
	\begin{itemize}
		\item Clustering: using more resources by running many instances of the application over a cluster of servers or instances; therefore if any server should fail, the reliability and availability of the system will not be compromised.
		\item For reading from and writing to the database, we will ensure that no parallel updates are possible through enforcing the use of a single object to stream all database transactions; thus reducing inaccuracy that would be a result of data redundancy.		 
		\item Use of more resources: This would heavily reduce system downtime, as a temporary server can be run while the other is maintained.
\end{itemize}
					
\end{enumerate}
