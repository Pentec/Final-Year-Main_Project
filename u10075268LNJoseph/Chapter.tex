\documentclass[a4paper,12pt]{report}
\addtolength{\oddsidemargin}{-1.cm}
\addtolength{\textwidth}{2cm}
\addtolength{\topmargin}{-2cm}
\addtolength{\textheight}{3.5cm}
\newcommand{\HRule}{\rule{\linewidth}{0.5mm}}
\makeindex



% define the title
\author{Pentec}
\begin{document}

% generates the title
\begin{titlepage}


\section{Chapter 1}


\section{Abstract}
This document is the Software User Manual (SUM) for the Patient Information Management System project\\ and was made according to the software engineering standard described in the tender proposal\\ provided by Professor Snyman. The Software User Manual (SUM) instructs how to install and use the\\
Patient Information Management System software. This project is part of the Software Engineering\\ Project course (COS301) at the University of Pretoria.  \\

\subsection{Change Log}
Document Title: Software User Manual\\
Version: 0.1.0\\
Document Status: conditionally approved\\
Version Date Author(s) Summary\\
0.0.1 29-05-2015 LN. Joseph Document creation\\


\subsection{Intended readership}
This document covers the use for the following users of the PIMS system:\\
\begin{description}
\item the system administrator
\item the project administrators
\item the medical staff
\item the usability test subjects
\end{description}

\subsection{Applicability}
This Software User Manual (SUM) applies to the PIMS software, version 0.1.\\


\subsection{Purpose}
The purpose of the SUM is to assist the user in installing and using the PIMS software.\\

	
\subsection{How to use this document}
	\begin{description}
\item How it is to be used:
\item[$\bullet$]  Chapter 2 - Overview of PIMS Software
\item[$\bullet$] Chapter 3 - tutorials
\item[$\bullet$] references of the complete PIMS software
\end{description}

	
\subsubsection{Problem Reporting}
Since the Pentec team will be dissolved after completion of the PIMS project, the\\
issue of problem reporting is left to the Administrator, Professor Snyman.\\
	


	\section{References}
\begin{description}
\item[$\bullet$] Department Obstetrics and Gynaecology Gynaecologic Oncology Unit Kalafong Hospital.
\end{description}


\end{titlepage}

\bibliography{myrefs}{} 
\bibliographystyle{ieeetr}
\end{document}
