
Model-View-Controller is an architectural pattern that divides software applications into three interconnected parts. \\ \par
The controller is responsible for translating requests in to responses (Nadel, 2012). The controller 'inserts' the response back into the view, which could be html or it's equivalents. \par
The view translates the response, from the controller, into a visual format for the client (Nadel, 2012). The view also sends requests to the controller. \par
The model's task is to maintain state and give methods for changing this state. Typically, this layer can be decomposed to:
\begin{itemize}
\item Service layer - provides high-level logic for dependent parts of an application.
\item Data access layer - services objects invoke this layer, which provides provides access to the persistence layer.
\item Value object layer - provides data-oriented representations of terminal nodes in the model hierarchy.
\end{itemize}
