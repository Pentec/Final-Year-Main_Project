Human Access Channels are all the various ways a user may interact with and access the PIMS.

\begin{enumerate}
	\item Mobile Phone:
	\begin{itemize}
		\item This mode of access will allow for better mobility and portability as the user can fill in information as they perform procedures. The user will have to use their own data to access the system.
	\end{itemize}
	\item Tablet:
	\begin{itemize}
		\item This mode of access, same as the mobile phone, will also for mobility and portability.
	\end{itemize}
	
	\item Desktop Computer:
	\begin{itemize}
		\item This will be the least common way for the user to access the PIMS, as the Hospital does not have a centralized computer.
	\end{itemize}
	\item Laptop:
	\begin{itemize}
		\item This mode of access is more portable than the desktop computer.
		\item The user will have to make use of a modem to access the Internet as the Hospital does not Wi-Fi access.
	\end{itemize}

\end{enumerate} 