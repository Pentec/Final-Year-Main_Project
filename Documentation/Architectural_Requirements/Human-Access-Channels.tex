Human Access Channels are all the various ways a user may interact with and access the PIMS system.

\begin{enumerate}
	\item Mobile Phone:
	\begin{itemize}
		\item This mode of access will allow for better mobility and portability, as the user can fill in information as they perform procedures.
		\item The user will have to use their own mobile data bandwidth to access the system, as the Hospital does not have Wi-Fi access.
	\end{itemize}
	\item Tablet:
	\begin{itemize}
		\item This mode of access, similar to the mobile phone, will also allow for better for mobility and portability.
	\end{itemize}
	
	\item Desktop Computer:
	\begin{itemize}
		\item This will be the least common way for the user to access the PIMS system, as the Hospital does not have a centralized computer.
		\item The user will have to make use of a personal modem to access the Internet as the Hospital does not have Wi-Fi access.
	\end{itemize}
	\item Laptop:
	\begin{itemize}
		\item This mode of access is more portable than the desktop computer.
		\item The user will have to make use of a personal modem to access the system, as the Hospital does not have Wi-Fi access.
	\end{itemize}

\end{enumerate} 