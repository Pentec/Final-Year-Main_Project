\subsection{Tactics and Strategies}
\subsubsection{Scalability and Performance}
	\begin{itemize}
		\item Optimize repeated processes.
		\item Reuse resources and results.
		\item Reduce contention by replicating frequently used resources.
		\item Clustering.
		\item Efficient use of storage.
		\item Load Balancing.
		\item Caching.
		\item Use REST (Representational State Transfer) to make Kalafong PIMS into a scalable web service.
	\end{itemize}	
\subsubsection{Security}
	\begin{itemize}
		\item Authenticate users by requesting user name and password when interaction with the system begins.
		\item Authorize users checking if a user has the rights to access and modify either data or services.
		\item Encryption to maintain confidentiality of data.
		\item Input Validation to detect malicious attacks.
		\item Auditing and logging for identifying and recovering from attacks.
	\end{itemize}	
\subsubsection{Usability}
	\begin{itemize}
		\item Usability is enhanced by giving the user feedback as to what the system is doing.
		\item Descriptive Error messages must be provide along with the necessary steps to address the errors.
		\item System must respond to actions performed by the user.
		\item Separate the user interface from the rest of the application using Model-View-Controller.
	\end{itemize}
\subsubsection{Integrability}
	\begin{itemize}
		\item Use modular programming to modularize the system. 
		\item Use REST (Representational State Transfer) to decouple Kalafong PIMS from other software that may need its services.
	\end{itemize}
\subsubsection{Maintainability}
	\begin{itemize}
		\item Use modular programming to modularize the system.
		\item Use object-oriented programming to sub divide the sub-system features.
	\end{itemize}
\subsubsection{Monitor-ability}
	\begin{itemize}
		\item Monitor-ability can be enhanced by using fault detection tactics:
		\begin{enumerate}
			\item Ping/echo: One component issues a ping and expects to receive back an echo, within a predefined time, from the component under scrutiny.
			\item Heartbeat: One component emits a message periodically and another component listens for it. If the heartbeat fails, the originating component is assumed to have failed and a fault correction component is notified.
			\item Exceptions: These are raised when an anomaly in a component occurs, encounter an exception when a fault is detected.
		\end{enumerate}
	\end{itemize}
\subsubsection{Reliability}
	\begin{itemize}
		\item Reliability can be enhanced by using fault recovery and preventions tactics:
		\begin{enumerate}
			\item Active redundancy: All redundant components respond to events concurrently. All redundant components will have the same state. When a fault occurs in the responding component, the system will switch to the next redundant component, minimizing downtime.
			\item Checkpoint/rollback: the states of the components will be recorded periodically or in response to certain events. When a fault occurs, the system should be restored to the previously consistent state or checkpoint.
			\item Transactions: The system should bundle several sequential steps in such a way that the entire bundle can be undone at once. 
		\end{enumerate}
	\end{itemize}
\subsubsection{Testability}
	\begin{itemize}
		\item Enhanced testability by recording the information that enters the system and using it as input into the test harness, and recording the output of the system components.
		\item Separating the interface from the implementation to enable substitution of implementations for various testing purposes.
		\item Creating a specialized testing interfaces to capture variable values to a system component and also seeing the output of the component in order to detect faults.
		\item The components can maintain useful information regarding its execution internally and then be viewed in the testing interface.
	\end{itemize}