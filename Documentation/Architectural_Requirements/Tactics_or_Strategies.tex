\subsubsection{Aspect Oriented programming} \label{sec:aop}
\begin{itemize}
		\item AOP is an interception mechanism that will allow for the addition of new services to the system.
This tactic will aid the following system functionalities:
		\begin{itemize}
			\item Logging (tactic: \ref{sec:logging})
		    \item System traceability
		\end{itemize}
\end{itemize}
\begin{itemize}
\item It will also address the following system requirements:
	\begin{itemize}
			\item \textbf{Auditability} (requirement: \ref{sec:auditability}) - logging functionality can be seamlessly injected across system services
			\item \textbf{Maintainability} (requirement: \ref{sec:maintainability}) - additional services can be added to the system
			\item \textbf{Integrability} (requirement: \ref{sec:integrability}) - AOP can make integration of other system components easier, particularly if those components would have to be inter-woven through the code.
	\end{itemize}
\end{itemize}

\subsubsection{Contracts-driven Development} \label{sec:contracts}
Service contracts were specified prior to the commencement of system implementation. The different pre and post-conditions pertaining to each functional requirement as well as the data structure constraints are enforced across the system services. The use of contracts addresses the following quality requirements:
\begin{itemize}
	\item \textbf{Testability} (requirement: \ref{sec:testability}) - Having specified the pre and post-conditions for each use case, any violation of a pre-condition will give a failing test and will throw an exception. Also, any failure to fulfill a post-condition will show a failure in a service of a the system.
	\item \textbf{Maintainability/Flexibility} (requirement: \ref{sec:maintainability}) - any system component that is replaced by another must be able to satisfy the same contract as the previous component
\end{itemize}


\subsubsection{Indexing} \label{sec:indexing}
Indexing of database objects will allow for faster retrieval of these objects; thus improving system performance as well as scalability.

\subsubsection{Templating} \label{sec:template}
Node.js offers many templating engines and these allow for:
\begin{itemize}
	\item A consistent user interface which improves system maintainability (requirement: \ref{sec:maintainability}) and usability (requirement: \ref{sec:usability}).
\end{itemize}

\subsubsection{Database Connection pools} \label{sec:pools}
Connection pooling will be used to achieve better system scalability (requirement: \ref{sec:scalability}) and performance (requirement: \ref{sec:performance})

\subsubsection{Logging} \label{sec:logging}
System logging will capture audit data and will help achieve system auditability (requirement: \ref{sec:auditability}), security (requirement: \ref{sec:security}) as well as maintainability (requirement: \ref{sec:maintainability}). The use of aspects will lend to the achievement of these requirements, as it will make the logging functionality flexible and maintainable.

\subsubsection{Client-side Rendering} \label{sec:clientrender}
By rendering pages on the client-side (in the browser), load is taken off the server to render pages and this is helps to improve system performance (requirement: \ref{sec:performance}) and scalability (requirement: \ref{sec:scalability}), as well as usability given that web pages will load quicker.


\subsubsection{Asynchronous processing} \label{sec:asyncprocess}
Asynchronous processing will be achieved through delaying tasks which may involve lengthy operations or tasks that have to wait for resources into the background to process. This will achieve better system scalability (requirement: \ref{sec:scalability}), performance(requirement: \ref{sec:performance}) and responsiveness, as other operations will not be blocked or held up.


\subsubsection{Framework for UI elements} \label{sec:uiframework}
By applying a dynamic Javascript UI component library, for all the user interface elements, the following will be achieved:
\begin{itemize}
		\item A dynamic user interface; thereby achieving system usability (requirement :\ref{sec:usability})
		\item System maintainability (requirement : \ref{sec:maintainability}), as components will be re-used
		\item System scalability \ref{sec:scalability} and performance \ref{sec:performance}, as these libraries will off-load system rendering to the client-side.
\end{itemize}	

\subsubsection{Dependency Injection}  \label{sec:depinject}
Dependency Injection will help in achieving the following:
\begin{itemize}
		\item Maintainability/flexibility (requirement :\ref{sec:maintainability}) - different system components can be replaced easily 
		\item Testability (requirement :\ref{sec:testability}) - dependency injection aids unit testing, as mock objects can be injected allowing for a system component to be tested as a stand-alone module.
		\item System components can be seamlessly applied in different environments.	
\end{itemize}
