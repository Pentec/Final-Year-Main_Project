\subsection{Platform}
\subsubsection{Node.js}
The system will be deployed in a Node.js environment. Node.js allows for queued inputs and since the system revolves around multiple inputs possibly going through at any particular time this provides a huge advantage over most other systems. Node.js also includes a wide range of modules through NPM (Node package manager). Some examples of these include Express, AngularJS, mongo, mongoose, and many more. Node.js processes programs asynchronously and thus is a no-interrupt driven language. This is as advantage because, as stated above, it allows for queued inputs. Node.js is written in JavaScript which means it is ubiquitous which is an obvious advantage.

Node.js relies heavily on callbacks to allow for synchronization which is an obvious hindrance for programmers that are not strong with recursion or callbacks. Although this con is outweighed by the advantages of the system.

\subsection{Framework}
\subsubsection{Express}
Express is a lightweight, high performance HTTP which supports URL configurable routing. This allows for a more professional look as well as improvements in maintainability and flexibility.
\subsection{AngularJS}
Angular is front-side development framework that applies the MVC architecture pattern. Angular speeds up client-side templating and allows for the programmers to write less code. Angular allows for dependency injection and is unit testing ready. This is a far better means than the traditional way of testing web applications by creating individual test pages that petition one component. This aids in usability, maintainability and performance.
\subsection{Broadway}
This framework creates a more flexible system as it allows the client to possibly implement further plug-ins to the web application.
\subsection{Crypto}
Crypto is an npm module that allows for multiple forms of encryption. It is lightweight and provides much needed security for the web application. It offers a way of encapsulating secure credentials over a secure HTTPS or HTTP connection.

\subsection{Express-Validator}
This npm package is an Express server middle-ware for the npm validator module. It is used for the purposes of form validation, particularly when making a post request to the server. This technology aids Usability, as it allows for the application to give feedback to a user concerning an input made.

\subsection{Node-mailer}
This npm package is for the notifications component of the system, allowing for the sending of email notifications to patients.
	
\subsection{Database}
\begin{itemize}
	\item MongoDB - is a NoSQL document store. MongoDB is used especially for applications that need store large volumes of data, which is mandatory for PIMS. \\
	MongoDB ensures that scalability, a critical requirement, is enforced. This is due to the highly cache-able persistence environment. Concurrency is also handled  via the locking mechanisms. Multiple read access and a single write operation is allowed.
\end{itemize}

\subsection{Unit Testing}
\subsubsection{Mocha}
\begin{itemize}
	\item Mocha was selected as our primary unit testing framework running for Node.js and the browser,
to assist in the simplification of asynchronous testing. \\
	Mocha tests run serially, allowing
 for flexible and accurate reporting, while mapping uncaught exceptions to the correct test 
 cases. We found that Mocha offered browser support which proves useful for our project as we are testing
our application on different browsers and a javascript API for running tests which assisted us in identifying our 
errors more visibly and made the framework much easier, both, to understand and use. Mocha is also very popular and 
so there is an online community of users offering assistance and suggestions, it is also very configurable. 
Mocha assisted us to describe our test suites.
\end{itemize}

\subsubsection{Chai}
\begin{itemize}
	\item We selected Chai as it is an assertion and expecting library for node and the browser given that we are
creating a web application and using javascript as our testing framework; Chai pairs them 
both very well. \\
Chai also works well with Mocha and other unit testing frameworks as we had
decided to use more than one to achieve readability for the output of our unit testing program. 
Chai assisted us to perform all kinds of assertions against our javascript code.
\end{itemize}

\subsubsection{Should}
\begin{itemize}
	\item Should is an expressive, readable, test framework agnostic, assertion library \\
	The main goals of the Should library is to be expressive and 
to be helpful. It helped keep our test code clean, and add meaning to error messages.
\end{itemize}
	

\subsection{Object Data Model}
\begin{itemize}
	\item Mongoose - is an object modelling environment for MongoDB and Node.js. It enforces contracts and structure via validation, provides connection pooling, which improves scalability. Automatic object to document mappings is also supported.
\end{itemize}

\subsection{Templating Engine}
\textbf{Jade:} The HTML code will be generated by the template engine Jade. This allows for minimal code and speeds up the entire process of writing HTML pages. Jade improves maintainability as the views are more readable and writeable.


\subsection{Application Servers}
\begin{itemize}
	\item TBA
\end{itemize}

\subsection{Web front-end}
The web front-end will make use of the following technologies:
\begin{itemize}
	\item Bootstrap
	\item HTML
	\item CSS
\end{itemize}

\subsection{Operating System}
\begin{itemize}
	\item Windows
	\item Possibly Android OS
\end{itemize}

\subsection{Dependency Management}
\begin{itemize}
	\item NPM
		\begin{itemize}
			\item This will be used to build the project and ensure that any component is not IDE-specific
			\item The details pertaining to each of the packages are within a package.json file				
		\end{itemize}
\end{itemize}

\subsection{Web Services}
\begin{itemize}
	\item SOAP-based
\end{itemize} 

\subsection{APIs}
\subsection{Other}
\begin{itemize}
	\item Heroku - A cloud based application platform.
\end{itemize}





