\subsection{Platform}
\subsubsection{Node.js}
The system will be deployed in a Node.js environment. Node.js allows for queued inputs and since the system revolves around multiple inputs possibly going through at any particular time this provides a huge advantage over most other systems. Node.js also includes a wide range of modules through NPM(Node package manager). Some examples of these include Express, AngularJS, mongo, mongoose, and many more. Node.js processes programs asynchronously and thus is a no-interrupt driven language. This is as advantage because, as stated above, it allows for queued inputs. Node.js is written in javascript which means it is ubiquitous which is an obvious advantage.

Node.js relies heavily on callbacks to allow for synchronization which is an obvous hinderance for programmers that are not strong with recursion or callbacks. Although this con is outweighed by the advantages of the system.


\subsection{Frameworks}
\begin{itemize}
		\item Express
		\item AngularJS
\end{itemize}

\subsection{Operating System}
\begin{itemize}
		\item Linux
\end{itemize}
	
	
\subsection{Databases}
\subsubsection{Relational Databases}
\begin{itemize}
	\item MySQL Database
\end{itemize}
	

\subsection{Object Relational Mappers}
\begin{itemize}
	\item JPQL
\end{itemize}

\subsection{Languages}
\subsubsection{Programming Languages}
\begin{itemize}
	\item JavaScript	
	\item Java
\end{itemize}

\subsubsection{Mark-up Languages}
\begin{itemize}
	\item HTML
\end{itemize}

/subsubsection{Template Engine}
\begin{itemize}
	\item Jade
\end{itemize}


\subsection{Application Servers}
\begin{itemize}
	\item GlassFish Server
	\item Tomcat
\end{itemize}

\subsection{Dependency Management}
\begin{itemize}
	\item Apache Maven
\end{itemize}

\subsection{Web Services}
\begin{itemize}
	\item SOAP-based
\end{itemize} 


\subsection{APIs}
\begin{itemize}
	\item Java Persistence API
\end{itemize}

\begin{itemize}
	\item JAX-RS RESTful web services
\end{itemize}

\begin{itemize}
	\item JAX-WS web service endpoints
\end{itemize}

\begin{itemize}
	\item Java Persistence API entities
\end{itemize}

\begin{itemize}
	\item The Java Database Connectivity API (JDBC)
\end{itemize}

\begin{itemize}
	\item The Java Persistence API
\end{itemize}

\begin{itemize}
	\item The Java EE Connector Architecture
\end{itemize}

\begin{itemize}
	\item The Java Transaction API (JTA)
\end{itemize}


\subsection{Others}
\begin{itemize}
	\item AJAX
\end{itemize}

\begin{itemize}
	\item Servlets
\end{itemize}

\begin{itemize}
	\item Java Server Pages
\end{itemize}


\begin{itemize}
	\item JavaServer Faces
\end{itemize}

\begin{itemize}
	\item JavaServer Faces Facelets
\end{itemize}

\begin{itemize}
	\item Enterprise JavaBeans (enterprise bean) components
\end{itemize}

\begin{itemize}
	\item Java EE managed beans
\end{itemize}


\section{Recommended Technologies}
\subsection{Databases}
\subsubsection{Object Relational Databases}
\begin{itemize}
	\item Postgresql
\end{itemize}

\subsubsection{NoSQL Databases}
\begin{itemize}
	\item Neo4j
	\item MongoDB
\end{itemize}

\subsection{Object Data Mappers}
To cater for the use of NoSQL Databases
\begin{itemize}
	\item Hibernate OGM
\end{itemize}
