%Scaleability
	\subsubsection*{Description}
	Scalability is an essential aspect of a system and is the ability of a system to be easily enlarged in order to accommodate a growing amount of work.
	\subsubsection*{Justification}
	 The PIMS should allow for hundreds of concurrent users, as such the system must be able to handle such a number without breaking down or reducing performance.
	\subsubsection*{Mechanism}
		\begin{enumerate}
			\item Strategy:
			\begin{itemize}
			\item Clustering: using more resources by running many instances of the application over a cluster of servers or instances, to ensure system resources are not strained by a high workload.
			
			\item Efficient use of storage: data storage can be efficiently used through compression of the data (reducing data size to make room for more) paging (ensuring that primary storage is used only for more crucial data) as well as de-fragmentation (organizing the data into continuous fragments and free more storage space).
			by ensuring that no data duplications occur, storage space can be conserved, thus the load on system resources will be reduced.
			\item Efficient persistence: through indexing and query optimization, the amount of system power used to persist a database will be reduced, as data retrieval will be quicker and costly queries will be done without, thus also reducing system load. In addition, connections can be grouped and accessed via a central channel in order to aid persistent storage to the database.
			
			\item Load Balancing: by spreading the systems load across time or across resources the load on the system can be distributed, therefore no system resource will be heavily strained. In the case that the limit for a server has been reached, a new instance or so will have to be created in order to handle the number of increasing requests. On the other hand, if the usage of a server is way below the capacity, the number of instances will have to be reduced.
			
			\item Caching: to ensure no duplication or repeated retrieval of frequent objects or queries; a separate module can facilitate caching; thus system resources will not be used up unnecessarily.
			 \end{itemize}

		\end{enumerate}