%Monitorability
	\subsubsection*{Description}
		Audititabily or Monitor-ability is a software review where one or more auditors/monitors who are not members of the software development and organization team conduct "An independent examination of the software product to assess compliance with specifications, standards, contractual agreements, functional requirements and other criteria according to the development specification. Software Monitor-ability and audiability is different from testing or peer reviews because they are done by personnel external to and independent of the software development organization.
	\subsubsection*{Justification}
	Although it is important for the PIMS to be monitorable, it is not crucial. It may be Monitored because of the following:
					\begin{itemize}
							\item Multiple and concurrent users, thus making it prone to various malfunctions.
							
							\item Form filling and Statistical quering needs to be done in real time at all times to ensure relevance of topics and subject matters.
							
							\item Sufficient feedback and updates of the site's state must be provided to the users.
							
							\item General software control and application usage..
						 \end{itemize}
	
	\subsubsection*{Mechanism}
		\begin{enumerate}
			\item Strategy:\\\\
			Auditability and monitor-ability will be achieved by allowing any third party software auditor or monitor support group reviewing the Buzz Space examining it specifically for the aspects of the functional requirements as given by the client. Information such as the systems state and processes in complaints with the specification given. One such auditor may be from a well-established organisation, for example Oracle’s PeopleSoft enterprise which is UP’s current used application.	
			
			
			 \item Pattern/Tools:\\\\
			 Tools like SMaRT, a workbench for reporting the monitor-ability of Service Level Agreements for software services such as Buzz Space may be used. SMaRT aims to clearly identify the service level the service level commitments established between service requesters and providers. This monitoring infrastructure can be used with mechanical support groups in the form of a SMaRT Workbench Eclipse IDE plug-in for reporting on the monitor-ability of Service Level Agreements.
			 
		\end{enumerate}
