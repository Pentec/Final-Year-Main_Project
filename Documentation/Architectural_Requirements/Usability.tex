%\subsection{Usability}
\subsubsection*{Description}
This ensures that a user will be able to use the system, with ease. The system should provide support to the user.
\subsubsection*{Justification}
The Patient information management system is user-oriented. How the users interact with the system is critical, and this should be done with little to no effort. The system should appear easy to use and should not, at any point, baffle the users. 
\subsubsection*{Mechanism}		
		 \begin{itemize}
		 	\item A tutorial on how the patient information management system works. A user can be initiated into the system, the first time they use it. Or they can enable the tutorial until they're familiar with the functionality.
		 	\item Enable the user to troubleshoot their problems. Frequently asked questions or frequent problems could assist with this aspect.  A user will be provided with predefined help options such that they will not need to contact the system's administrator, for assistance.
		 	\item Provide descriptive headings that make navigation easier. Headings should not be ambiguous. A user should know what to expect when they select a certain heading.
		 	\item Error signals should be displayed to the user, if some user-inflicted error occurs. The necessary steps to rectify this problem must be provided.
		 	\item A user should be able to undo their action, should they be aware of their mistake.
			\item Model-View-Controller: This separates the user interface from the rest of the system (Bass and John). A user should only interact with a simple interface that was designed for them. This is describable for patient information management system because the users don't necessary have an adept understanding of the lower   levels of the system.
		\end{itemize}