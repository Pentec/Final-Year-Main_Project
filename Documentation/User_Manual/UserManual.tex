\documentclass[a4paper]{article}

\title{User Manual
\\COS 301 Main Project
\\Gynaecological Patient Information Management System
\\Team :Pentec
\\Version 1.0}

\author{Ruth Ojo 12042804
\\Liz Joseph 10075268
\\Trevor Austin 11310856
\\Maria Qumayo 29461775
\\Lindelo Mapumulo 12002862
\\ University of Pretoria}

\date{1 June 2015}

\begin{document}

\maketitle
% No page number to cover page
\pagenumbering{gobble}
\newpage
% start page numbering
\pagenumbering{arabic}

% Generate Table of Contents

\tableofcontents
\newpage


\section{Abstract}
This document is the Software User Manual (SUM) for the Patient Information Management System project and was made according to the software engineering standard described in the tender proposal provided by Professor Snyman. The Software User Manual (SUM) instructs how to install and use the
Patient Information Management System software. This project is part of the Software Engineering Project course (COS301) at the University of Pretoria.\\

\newpage

\section{Chapter 1}



\subsection{Change Log}
Document Title: Software User Manual\\
Version: 0.1.0\\
Document Status: conditionally approved\\
Version Date Author(s) Summary\\
0.0.1 29-05-2015 LN. Joseph Document creation\\


\subsection{Intended readership}
This document covers the use for the following users of the PIMS system:\\
\begin{description}
\item the system administrator
\item the project administrators
\item the medical staff
\item the usability test subjects
\end{description}

\subsection{Applicability}
This Software User Manual (SUM) applies to the PIMS software, version 0.1.\\


\subsection{Purpose}
The purpose of the SUM is to assist the user in installing and using the PIMS software.\\

	
\subsection{How to use this document}
\begin{description}
\item How it is to be used:
\item[$\bullet$]  Chapter 2 - Overview of PIMS Software
\item[$\bullet$] Chapter 3 - tutorials
\item[$\bullet$] references of the complete PIMS software
\end{description}

	
\subsubsection{Problem Reporting}
Since the Pentec team will be dissolved after completion of the PIMS project, the
issue of problem reporting is left to the Administrator, Professor Snyman.\\
	

\section{Chapter 2}
\subsubsection{Overview}
The software is used by doctors and medical staff to create patient forms and document patient information and to query the forms for statistical information. 
\\

\section{Chapter 3}
\subsection{Tutorial}
\subsubsection{Running the Software}
\begin{enumerate}
  \item Website page
  \begin{enumerate}
  \item Establish an internet connection
    \item Search for website in web browser
    \item Log into PIMS system with given authentication codes
  \end{enumerate}
  \item Mobile Application
   \begin{enumerate}
   \item Establish an internet connection
    \item Search for website in web browser
    \item Log into PIMS system with given authentication codes
  \end{enumerate}
  \item Tablet or other
   \begin{enumerate}
\item Establish an internet connection  
    \item Search for website in web browser
    \item Log into PIMS system with given authentication codes
  \end{enumerate} \ldots
\end{enumerate}

\subsubsection{Shutting down the website}
Contact your service provider(hosting site) to pull down the software


\section{References}
\begin{description}
\item The login, forms, user and statistics operations are described in this chapter.
\end{description}


\subsection(User)
	  
\subsubsection add user
\begin{description}
\item [Functional Description] This operation adds a user to the PENTEC-PIMS database.
\item [Formal Description] \hfill
\begin{itemize}
	\item Syntax: add user (username,surname,password,staff type,department,email address,user right) as [Users] [A schema to save the details of all medical staff that can access the system]\\
	\item Parameters:
		\begin{itemize}
			\item [schema] (Required when chosen) : Users Schema\\
			\item [pentec\_pims] (Required) : This is the name of the database in mongoose that we are using\\
			\item [details](Required):All the above mentioned details in syntax are important to add a user\\
		 \end{itemize}
\end{itemize}
\item[Examples]\hfill
\begin{itemize}
	\item add user John Doe, Medical Intern, User right 2, Gynaecologist
	\item add user url:http://kalafongpims.herokuapp.com/addUser
	\item URL : :http://kalafongpims.herokuapp.com as Medical Staff This is a medical research dataset PENTEC Software User Manual 0.1.0 14
\end{itemize}
\item[Possible errors]\hfill
	\begin{itemize}
	\item You do not have the login creditials
	\item A user with name [username]already exists
	\item You don't have the role of admin
	\end{itemize}
\item[Solutions]\hfill
	\begin{itemize}
	\item Go to admin(Dr Snyman) and request he add you to the database of users.
	\item Register with your already given details. No duplicates allowed.
	\item  Go to admin(Dr Snyman) and request he make you admin.
	\end{itemize}
\item[Related operations] remove user
\end{description}
	  
\subsubsection edit profile
\begin{description}
\item[Functional Description] This operation edits a users profile and saves it to the PENTEC-PIMS database.
\item[Formal description]\hfill
\begin{itemize}
	\item Syntax: edit user (username,surname,password,confirm password,staff type,department,email address,user right) as [Users] [A schema to save the details of all medical staff that can access the system]\\
	\item Parameters:
		\begin{itemize}
			\item [schema] (Required when chosen) : Users Schema
			\item [pentec\_pims] (Required) : This is the name of the database in mongoose that we are using.
			\item [details] (Required) :All the above mentioned details in syntax are important to complete edit profile.
		\end{itemize}
\end{itemize}

\item[Examples]\hfill
\begin{itemize}
	\item edit user John Doe, Medical Intern, User right 2, Gynaecologist
	\item edit user url:http://kalafongpims.herokuapp.com/editProfile
	\item URL : :http://kalafongpims.herokuapp.com as Medical Staff This is a medical research dataset PENTEC Software User Manual 0.1.0 14
\end{itemize}

\item[Possible errors]\hfill
\begin{itemize}
	\item You do not have the login creditials to log into the system
	\item Passwords don't match
	\item You don't have the role of admin
	\item You do not appear on the system
\end{itemize}

\item[Solutions]\hfill
\begin{itemize}
	\item Go to admin(Dr Snyman) and request he add you to the database of users.
	\item Register with your already given details sent to your email. No duplicates allowed.
	\item Re-enter your password
	\item  Go to admin(Dr Snyman) and request he make you admin.
\end{itemize}
\item[Related operations] add user
\end{description}
		   
\subsubsection password
\begin{description}
\item[Functional description] This operation changes your password on the PENTEC-PIMS database.
\item[Formal description]\hfill
\begin{itemize}
	\item Syntax:password (confirm password, password) as [Users] [A schema to save the details of all medical staff that can access the system]\\
	\item Parameters:
	\begin{itemize}
		\item [schema] (Required when chosen) : Users Schema
		\item [pentec\_pims] (Required) : This is the name of the database in mongoose that we are using.
		\item [details] (Required) :All the above mentioned details in syntax are important to complete password.
	\end{itemize}
\end{itemize}

\item[Examples]\hfill
\begin{itemize}
	\item password mysecretpassword, mysecretpassword
	\item add user url:http://kalafongpims.herokuapp.com/editProfile
	\item URL : :http://kalafongpims.herokuapp.com as Medical Staff This is a medical research dataset
	PENTEC Software User Manual 0.1.0 14
\end{itemize}
\item[Possible errors]\hfill
\begin{itemize}
	\item You do not have the login creditials
	\item You don't have the role of admin
	\item Passwords dont match
\end{itemize}
\item[Solutions]\hfill
\begin{itemize}
	\item Go to admin(Dr Snyman) and request he add you to the database of users.
	\item  Go to admin(Dr Snyman) and request he make you admin.
	\item Re-enter your password carefully.
\end{itemize}
\item[Related operations] none
\end{description}

\subsubsection list form
\begin{description}
\item[Functional description] This operation list all the available forms in the PENTEC-PIMS database.
\item[Formal description]\hfill
\begin{itemize}
	\item Syntax: list form (form name) as [Forms] [A schema to save the details of all medical forms in the system]\\
	\item Parameters:
	\begin{itemize}
		\item [schema] (Required when chosen) : Forms Schema
		\item [pentec\_pims] (Required) : This is the name of the database in mongoose that we are using.
		\item [details] (Required) :All the above mentioned details in syntax are important to complete list form.
	\end{itemize}
\end{itemize}

\item[Examples]\hfill
\begin{itemize}
	\item list form Gynaecology Form
	\item list form url:http://kalafongpims.herokuapp.com/forms
	\item URL : :http://kalafongpims.herokuapp.com as Medical Staff This is a medical research dataset
	PENTEC Software User Manual 0.1.0 14
\end{itemize}

\item[Possible errors]\hfill
\begin{itemize}
	\item You do not have the login creditials
	\item The form you are looking for does not exist
\end{itemize}

\item[Solutions]\hfill
\begin{itemize}
	\item Go to admin(Dr Snyman) and request he add you to the database of users.
	\item Go to admin(Dr Snyman) and request he create the form.
\end{itemize}
\item[Related operations] none

\end{description}
	      
	  
\subsubsection add form
\begin{description}
\item[Functional description] This operation allows you to add a new form using the form builder to the PENTEC-PIMS database.
\item [Formal description]\hfill
\begin{itemize}
	\item Syntax: add form (data, form name) as [Forms] [A schema to save the details of all medical staff that can access the system]\\
	\item Parameters:
	\begin{itemize}
		\item [schema] (Required when chosen) : Forms Schema
		\item [pentec\_pims] (Required) : This is the name of the database in mongoose that we are using.
		\item [details] (Required) :All the above mentioned details in syntax are important to complete add form.
	\end{itemize}
\end{itemize}
\item[Examples]\hfill
\begin{itemize}
	\item add form Checkout form, JSON Object
	\item add form url:http://kalafongpims.herokuapp.com/formbuilder
	\item URL : :http://kalafongpims.herokuapp.com as Medical Staff This is a medical research dataset
	PENTEC Software User Manual 0.1.0 14
\end{itemize}

\item[Possible errors]\hfill
\begin{itemize}
	\item You cannot find the form you created
\end{itemize}

\item[Solutions]\hfill
\begin{itemize}
	\item It should now be listed in the dropdownlist on the page forms
\end{itemize}
\item[Related operations] list form
\end{description}
	      
	      
\subsubsection save form
\begin{description}
\item[Functional description] This operation allows you to save a form to the PENTEC-PIMS database.
\item [Formal description]\hfill
\begin{itemize}
	\item Syntax: save form (data, form name) as [Forms] [A schema to save the details of all forms created in the system]\\
	\item Parameters:
	\begin{itemize}
		\item [schema] (Required when chosen) : Forms Schema
		\item [pentec\_pims] (Required) : This is the name of the database in mongoose that we are using.
		\item [details] (Required) :All the above mentioned details in syntax are important to complete save form.
	\end{itemize}
\end{itemize}

\item[Examples]\hfill
\begin{itemize}
	\item save form discharge form, JSON Object
	\item save form url:http://kalafongpims.herokuapp.com/formbuilder
	\item URL : :http://kalafongpims.herokuapp.com as Medical Staff This is a medical research dataset
	PENTEC Software User Manual 0.1.0 14
\end{itemize}

\item[Possible errors]none
\item[Solutions] none
\item [Related operations] list form, add form
\end{description}

\subsubsection send notification
\begin{description}
\item[Functional description] This operation allows you to notify a patient of their follow up via email.
\item[Formal description]\hfill
\begin{itemize}
	\item Syntax: add notification (username, email, message, date) as [Users] [A schema to save the details of all medical staff that can access the system]\\
	\item Parameters:
	\begin{itemize}
		\item [schema] (Required when chosen) : Users Schema
		\item [pentec\_pims] (Required) : This is the name of the database in mongoose that we are using.
		\item [details] (Required) :All the above mentioned details in syntax are important to complete send notification.
	\end{itemize}
\end{itemize}
\item[Examples]\hfill
\begin{itemize}
	\item add notification John Doe,john@gmail.com, "John Please come for your checkup",2015
	\item add notification url:http://kalafongpims.herokuapp.com/sendNotification
	\item URL : :http://kalafongpims.herokuapp.com as Medical Staff This is a medical research dataset
	PENTEC Software User Manual 0.1.0 14
\end{itemize}

\item[Possible errors]\hfill
\begin{itemize}
	\item You cant find the users name
	\item User does not have an email address
\end{itemize}


\item[Solutions]\hfill
\begin{itemize}
	\item Patient does not exist in the system
	\item Notification cannot be sent to user
\end{itemize}


\item[Related operations] remove user
\end{description}
	   
\subsubsection list department
\begin{description}
\item[Functional description] This operation lists all the departments on the splash screen from the PENTEC-PIMS database.
\item[Formal description]\hfill
\begin{itemize}
	\item Syntax: list department (name of department, link) as [Departments] [A schema to save the details of all departments in the system and display them on the splash screen]\\
	\item Parameters:
	\begin{itemize}
		\item [schema] (Required when chosen) : Departments Schema
		\item [pentec\_pims] (Required) : This is the name of the database in mongoose that we are using.
		\item [details] (Required) :All the above mentioned details in syntax are important to complete list user.
	\end{itemize}
\end{itemize}

\item[Examples]\hfill
\begin{itemize}
	\item list department Gynaecology, www/d/
	\item list department url:http://kalafongpims.herokuapp.com/splash
	\item URL : :http://kalafongpims.herokuapp.com as Medical Staff This is a medical research dataset
	PENTEC Software User Manual 0.1.0 14
\end{itemize}
\item[Possible errors] None
\item[Solutions] None
\item[Related operations] none
\end{description}
	      
\subsubsection exit
\begin{description}
\item[Functional description] This operation adds a user to the PENTEC-PIMS database.
\item[Formal description]\hfill
\begin{itemize}
	\item Syntax: add user (none) as [none] [none]\\
	\item Parameters:
	\begin{itemize}
		\item [schema] (Required when chosen) : Users Schema
		\item [pentec\_pims] (Required) : This is the name of the database in mongoose that we are using.
		\item [details] (Required) :All the above mentioned details in syntax are important to complete exit.
	\end{itemize}
\end{itemize}

\item[Examples]\hfill
\begin{itemize}
	\item exit [press logout button]
	\item exit url:http://kalafongpims.herokuapp.com/home
	\item URL : :http://kalafongpims.herokuapp.com as Medical Staff This is a medical research dataset
	PENTEC Software User Manual 0.1.0 14
\end{itemize}

\item[Possible errors] none
\item[Solutions] none
\item [Related operations] login
\end{description}

\subsection{Login}
\subsubsection{authenticate} 
\begin{description}
\item[Functional Description] This function authenticates a user and logs them into the system.
\item[Formal Description]\hfill
\begin{itemize}
	\item Syntax: authenticate(<username>, <password>, <callback>)\\
	\item Parameters:
		\begin{itemize}
			\item <username> (Required): This is the username of the user, it will usually be the user's name or a unique number assigned to them.
			\item <password> (Required): This is the password of the user and is used to authenticate the user.
			\item <callback>(Opitional): This is the callback function and is used to keep the processes synchronous.
		\end{itemize}
\end{itemize}
\item[Examples]\hfill
\begin{itemize}
	\item authenticate("John", 1234)
	\item authenticate("john", 1234, thisIsAFunction(){})
\end{itemize}
\item[Possible Errors \&  Solutions]\hfill
	\begin{itemize}
		\item Problem: Your login does not exist.
		\item Solution: Contact supervisor.
		\item Problem: Your password is incorrect.
		\item Solution: Contact supervisor.
		\item Problem: You are trying to access an admin page but it takes you to a regular user page.
		\item Solution: Request supervisor to change your rank.
	\end{itemize}
\item [Related operations]	checkAdmin
\end{description}

\subsubsection{check admin}
\begin{description}
\item[Functional Description] Determines if the user is an administrator.
\item[Formal Description]\hfill
\begin{itemize}
	\item Syntax: checkAdmin(<username>, <password>, <callback>)\\
	\item Parameters:
		\begin{itemize}
			\item <username> (Required): This is the username of the user, it will usually be the user's name or a unique number assigned to them.
			\item <password> (Required): This is the password of the user and is used to authenticate the user.
			\item <callback>(Opitional): This is the callback function and is used to keep the processes synchronous.
		\end{itemize}
\end{itemize}
\item[Examples]\hfill
\begin{itemize}
	\item checkAdmin("John", 1234)
	\item checkAdmin("john", 1234, thisIsAFunction(){})
\end{itemize}
\item[Possible Errors \& Solutions]
\begin{itemize}
	\item Problem: You are trying to access an admin page but takes you to a regular user page.
	\item Solution: Request supervisor to change your rank.
\end{itemize}
\item[Related operations] authenticate
\end{description}

\subsection{Statistics}

\subsubsection{get procedure}
\begin{description}
\item[Functional Description] This function gets certain procedures based on its selectors and returns them in a format which can be converted into a graph.
\item[Formal Description]/hfill
\begin{itemize}
	\item Syntax: getProcedure(<procedure>, <selectors>, <callback>)\\
	\item Parameters:
		\begin{itemize}
			\item <procedure>(Required) This is the procedure that has been selected.
			\item <selectors>(Required) These are the selectors put in place to customize the procedures that are selected.
			\item <callback>(Optional) This isi the callback function and is used to keep the processes synchronous.
		\end{itemize}
\end{itemize}
\item[Examples]\hfill
\begin{itemize}
	\item getProcedure("Hysterectomy", {startDate : "10/10/2014", endDate: "10/10/2015"})
	\item getProcedure("Hysterectomy", {startDate : "10/10/2014", endDate: "10/10/2015"}, callback)
\end{itemize}
\item[Possible Errors \& Solutions]
\begin{itemize}
	\item Problem: Invalid procedure
	\item Solution: Choose a valid procedure
	\item Problem: Invalid selectors
	\item Solution: Choose a valid selector.
\end{itemize}
\item[Related operations] \hfill
\begin{itemize}
	\item get patients
	\item get doctor
\end{itemize}
\end{description}

\subsubsection{get patients}
\begin{description}
\item[Functional Description] This function gets admission based data on patients and certain selectors and returns them in a format which can be converted into a graph.
\item[Formal Description]/hfill
\begin{itemize}
	\item Syntax: getPatients(<selectors>, <callback>)\\
	\item Parameters:
		\begin{itemize}
			\item <selectors>(Required) These are the selectors put in place to customize the procedures that are selected.
			\item <callback>(Required) This isi the callback function and is used to keep the processes synchronous.
		\end{itemize}
\end{itemize}
\item[Examples]\hfill
\begin{itemize}
	\item getProcedure({startDate : "10/10/2014", age: 25, endDate: "10/10/2015"})
	\item getProcedure({startDate : "10/10/2014", age: 25, endDate: "10/10/2015"}, callback)
\end{itemize}
\item[Possible Errors \& Solutions]
\begin{itemize}
	\item Problem: Invalid selectors
	\item Solution: Choose a valid selector.
\end{itemize}
\item[Related operations] \hfill
\begin{itemize}
	\item get procedure
	\item get doctor
\end{itemize}
\end{description}

\subsubsection{get doctor}

\begin{description}
\item[Functional Description] This function gets data about specific doctors based on its selectors and returns them in a format which can be converted into a graph.
\item[Formal Description]/hfill
\begin{itemize}
	\item Syntax: getProcedure(<doctor>, <selectors>, <callback>)\\
	\item Parameters:
		\begin{itemize}
			\item <doctor>(Required) This is the doctor that has been selected.
			\item <selectors>(Required) These are the selectors put in place to customize the procedures that are selected.
			\item <callback>(Optional) This isi the callback function and is used to keep the processes synchronous.
		\end{itemize}
\end{itemize}
\item[Examples]\hfill
\begin{itemize}
	\item getProcedure("Dr. Snyman", {startDate : "10/10/2014", procedure: "all", endDate: "10/10/2015"})
	\item getProcedure("Dr. Snyman", {startDate : "10/10/2014", procedure: "all" ,endDate: "10/10/2015"}, callback)
\end{itemize}
\item[Possible Errors \& Solutions]
\begin{itemize}
	\item Problem: Doctor does not exist
	\item Solution: Determine whether Doctor exists, if they do contact supervisor.
	\item Problem: Invalid selectors
	\item Solution: Choose a valid selector.
\end{itemize}
\item[Related operations] \hfill
\begin{itemize}
	\item get patients
	\item get procedures
\end{itemize}
\end{description}



\end{document}

